%\documentclass{beamer}
%\documentclass[slidestop,usepdftitle=false]{beamer}
\documentclass[english,10pt,table]{beamer}
%\documentclass[english,10pt,table,handout]{beamer}

\input{style.tex}
\lecture[0]{Khảo sát kết quả của bài tập online cho phép nộp bài nhiều lần}{lecture-text}

\usepackage{pifont}
% Symbol definitions for these lists
\newcommand{\DingListSymbolA}{43}
\newcommand{\DingListSymbolB}{243}
\newcommand{\DingListSymbolC}{224}
\newcommand{\DingListSymbolD}{219}

% Boxed equation
\definecolor{LightYellow}{rgb}{1.,1.,.9}
\definecolor{LightRed}{rgb}{1.,.6,.6}


%%ensembles de nombres
\def\NP{$\mathcal{NP}$}
\def\N{\mathbb{N}}
\def\Z{\mathbb{Z}}
\def\R{\mathbb{R}}
\def\Q{\mathbb{Q}}

%\date[]{~~}
%%%%%%%%%%%%%%%%%%%%%%%%%%%%%%%%%%%%%%%%%%%%%%%%%%%%%%%%%%%%%%%%%%%%%
%%%%%%%%%%%%%%%%%%%%%%%%%%%%%%%%%%%%%%%%%%%%%%%%%%%%%%%%%%%%%%%%%%%%%
\begin{document}
\frame{
\selectlanguage{vietnamese}
  \maketitle
}


%%%%%%%%%%%%%%%%%%%%%%%%%%%%%%%%%%%%%%%%%%%%%%%%%%%%%%%%%%%%%%%%%%%%%
%%%%%%%%%%%%%%%%%%%%%%%%%%%%%%%%%%%%%%%%%%%%%%%%%%%%%%%%%%%%%%%%%%%%%
%\section[Plan]{}
%\setcounter{tocdepth}{1}
\frame{ \tableofcontents}
%\setcounter{tocdepth}{5}
% to display left summary deeper and plan slide juste display section: add command \setcounter{tocdepth}{1} and then \setcounter{tocdepth}{10}  recompile twice or more again 

%%%%%%%%%%%%%%%%%%%%%%%%%%%%%%%%%%%%%%%%%%%%%%%%%%%%%%%%%%%%%%%%%%%%%
%%%%%%%%%%%%%%%%%%%%%%%%%%%%%%%%%%%%%%%%%%%%%%%%%%%%%%%%%%%%%%%%%%%%%
\section{File 1}
\subsection{Bài 1}
\frame
{
  \frametitle{Bài 1: Xác định số lượng sinh viên}	
\begin{block}{Đâu nào là một mệnh đề}
Xác định giá trị chân lý của mệnh đề đó và tìm phủ định của nó.
\begin{enumerate}[a]
	\item Miami là thủ phủ của bang Florida.
	\item 1+1=2.
	\item Hãy tìm giá trị của $x$ nếu biết $x+2=5$
	\item Đây là đường một chiều.
	\item Bây giờ là mấy giờ?
	\item Đất đỏ bazane trồng cây rất tốt.
	\item $x+y=y+z$ nếu $x=z$.
	\item Hôm nay là thứ năm
	\item Không có ô nhiễm ở Bảo Lộc, nhưng có nhiều ô nhiễm ở Đà Lạt.
	\item Mùa hè ở thành phố Seville thì nóng và nắng.
\end{enumerate}
\end{block}
}

\frame
{
  \frametitle{Bài 1: Xác định số lượng sinh viên}	
\begin{block}{}
Chứng minh các mệnh đề sau đây tương đương: 
\begin{enumerate}[a]
	\item $\lnot (p \leftrightarrow q) $ và $\lnot p \leftrightarrow q$ 
	\item $(p \to q) \land (p \to r)$ và $p \to (q \land r)$  
	\item $(p \to r) \land (q \to r)$ và $(p \lor q) \to r$
	\item $(p \to q) \lor (p \to r)$ và $p \to (q \lor r)$ 
	\item $\lnot p \to (q \to r)$ và $q \to (p \lor r)$ 
	\item $p \leftrightarrow q$ và $(p \to q) \land (q \to p)$ 
\end{enumerate}
\end{block}
}

\frame
{
  \frametitle{Proving methods}	
\begin{block}{}
Gọi $P$, $Q$ là các mệnh đề: 
\begin{itemize}
	\item $P$: ``Minh giỏi Toán''
	\item $Q$: ``Minh yếu Anh văn''
\end{itemize}
Hãy viết lại các mệnh đề sau dưới dạng hình thức trong đó sử dụng các phép nối. \\
(Giả sử đảo nghĩa của học 'giỏi' là học 'yếu'.)
\begin{enumerate}[a]
\item Minh giỏi Toán nhưng yếu Anh văn
\item Minh yếu cả Toán lẫn Anh văn
\item Minh giỏi Toán hay Minh vừa giỏi Anh văn nhưng vừa yếu Toán
\item Nếu Minh giỏi Toán thì Minh giỏi Anh văn
\item Minh giỏi Toán và Anh văn hay Minh yếu Toán nhưng giỏi Anh
\end{enumerate}

\end{block}
}
%%%%%%%%%%%%%%%%%%%%%%%%%%%%%%%
\subsection{Bài 2}
\frame
{
  \frametitle{Predicate logic}	
\begin{block}{}
Cho vị từ $N(x)$ ``$x$ đã từng đi chơi Đà Lạt'' với tập vũ trụ là toàn bộ sinh viên trong lớp Cấu trúc rời rạc. 
Hãy phát biểu các vị từ sau:
\begin{enumerate}[a]
	\item $\exists x N(x)$ 
	\item $\forall x N(x)$ 
	\item $\neg \exists x N(x)$ 
 \item $\exists x \neg N(x) $ 
	\item $\neg \forall x N(x)$ 
	\item $\forall x \neg N(x)$
\end{enumerate}
\end{block}
}

%%%%%%%%%%%%%%%%%%%%%%%%%%%%%%%
\subsection{Bài 4}
\frame
{
  \frametitle{Proving methods}	
\begin{block}{}
Show that if $n$ is an integer and $n^3 + 2015$ is odd, then $n$ is even using
\begin{enumerate}[a]
\item a proof by contraposition.
\item a proof by contradiction.
\end{enumerate}
\end{block}
}


%%%%%%%%%%%%%%%%%%%%%%%%%%%%%%%%%%%%%%%%%%%%%%%%%%%%%%%%%%%%%%%%%%%%%
%%%%%%%%%%%%%%%%%%%%%%%%%%%%%%%%%%%%%%%%%%%%%%%%%%%%%%%%%%%%%%%%%%%%%
\subsection{Bài 5}
\frame{}
\subsection{Bài 7}
\frame{}
\subsection{Bài 9}
\frame{}
\end{document}

